\documentclass[11pt]{jsarticle}

%\usepackage{color}
%\usepackage{xcolor}
%\usepackage{textcomp}
%\usepackage[utf8]{inputenc}
\usepackage{graphicx}
%\usepackage{comment}
\usepackage{listings, jlisting}
%\usepackage{listings-golang}

\title{電気電子計算工学及演習 課題3}
\author{三軒家 佑將(さんげんや ゆうすけ) \\ 3回生 \\ 1026-26-5817 \\ a0146089}
\date{}

\newcommand{\fg}[3]{ % \fg{label}{path}{caption},
	\begin{figure}
		\includegraphics[width=\textwidth]{#2}
		\caption{#3}
		\label{#1}
	\end{figure}
}

\newcommand{\tab}[3]{ % \tab{label}{csv-path}{caption}
	\begin{table}[htb]
  		\begin{center}
			%\input{#2}
			\csvautotabular{#2}
    		\caption{#3}
			\label{#1}
  		\end{center}
	\end{table}
}

\newcommand{\code}[2]{
	\lstinputlisting[title={ソースコード: #2(#1)}]{ ../#2 }
}

\newcommand{\fr}[1]{図\ref{#1}}
\newcommand{\tr}[1]{表\ref{#1}}

\begin{document}
    \maketitle
    
    \section{プログラムの説明}
        \subsection{概要}
            本レポートにおいては、プログラム言語としてRubyを採用した。

            プログラムを実行する手順は、以下のとおりである。以下の手順に従うことで、課題3.1, 3.2, 3.3, 3.4の4つ全てに関して、結果をグラフにした画像がgraphsディレクトリ以下に出力される。
            \begin{lstlisting}[language=bash]
    ?> cd src
    ?> bundle install --path vendor/bin
    ?> bundle exec ruby main.rb 2> /dev/null
            \end{lstlisting}
            二行目で、依存ライブラリのインストールを行っている。また、三行目は、プログラムを実行するコマンドである。
            エラー出力を/dev/nullにリダイレクトしているのは、線形回帰に用いたライブラリの警告メッセージを表示しないためである。
            リダイレクトを行わなくても、プログラムは問題なく実行される。

        \subsection{各機能・関数の説明}
            プログラムを作成するにあたって、見通しを良くするために、プログラムを複数のファイルに分割している。
            ここでは、各ファイルごとに、そのファイルの担う機能と、そのファイル内にある関数の機能などについて簡単に説明する。

            各関数の詳しい使用方法などは、プログラム内のコメントにて示したので、そちらも参照されたい。

            \subsubsection*{calculation.rb}
                各課題の数値計算を行なう部分のうち、共通する部分を切り出したものである。calculate関数とall\_calculations関数を含む。

                calculate関数は、渡された各種パラメーターと、渡されたブロックで表されたアルゴリズムに基づいて、数値計算を行なう。

                all\_calculations関数は、渡された各種パラメーターと、渡されたブロックで表されたアルゴリズムに基づいて、calculate関数を内部で複数回呼び出し、課題3.1と3.2に示された各種数値計算を行なう。

            \subsubsection*{vector.rb}
                一次元のベクトルを表すMyVectorクラスを定義している。

                MyVectorクラスは、Rubyの組み込みクラスであるArrayクラスを継承して定義した。
                Arrayクラスの機能に加えて、ベクトル間の加算(+)・減算(-)と、ベクトル-スカラー間の乗算(*)・除算(/)を定義している。
                また、MyVectorクラスには、ベクトルの大きさ(二乗和平方根)を求めるnormメソッドと、要素の合計を求めるsumメソッドを定義した。
                さらに、MyVectorクラスのインスタンスを簡単に生成するために、Arrayクラスに、to\_vメソッドを追加した。

            \subsubsection*{plot.rb}
                グラフを描画し、ファイルに出力する機能を担う。gnuplotのラッパーを利用している。

                draw\_graphs関数に各種パラメーターを渡すことで、graphsディレクトリ以下にグラフの画像が出力される。save\_graphs関数は、draw\_graphs関数に呼び出され、実際にグラフを出力する処理を行なう。

            \subsubsection*{least\_square.rb}
                線形回帰を行って、一次関数の係数を求める機能を担う。statsampleというライブラリを利用している。

                least\_square関数が定義されており、xの配列とyの配列を与えると、その2つのデータの間に$y=a+bx$の関係があると考え、$a$と$b$の値を求める。

            \subsubsection*{main.rb}
                上記で述べた関数を利用して、実際にオイラー法・ホイン法・四次のルンゲ-クッタ法にて、微分方程式の数値解を求める。

    \section{課題3.1}
        % アルゴリズムの解説
        % 結果(グラフを貼り付ける、傾きに関しては標準出力からの出力であることを明記する)
        % 考察(計算方法に関する考察、得られた結果の精度、うまく行かなかったと思われる点、その他、気付いた点や考えたことについて)
    
    \section{課題3.2}
        % アルゴリズムの解説
        % 結果(グラフを貼り付ける、傾きに関しては標準出力からの出力であることを明記する)
        % 考察(計算方法に関する考察、得られた結果の精度、うまく行かなかったと思われる点、その他、気付いた点や考えたことについて)

    \section{課題3.3}
        % アルゴリズムの解説
        % 結果(グラフを貼り付ける、傾きに関しては標準出力からの出力であることを明記する)
        % 考察(計算方法に関する考察、得られた結果の精度、うまく行かなかったと思われる点、その他、気付いた点や考えたことについて)

    \section{課題3.4}
        % アルゴリズムの解説
        % 結果(グラフを貼り付ける、傾きに関しては標準出力からの出力であることを明記する)
        % 考察(計算方法に関する考察、得られた結果の精度、うまく行かなかったと思われる点、その他、気付いた点や考えたことについて)

    \section{付録}
        % 実行結果のコピー
\end{document}
