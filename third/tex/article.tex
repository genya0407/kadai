\documentclass[11pt]{jsarticle}

%\usepackage{color}
%\usepackage{xcolor}
%\usepackage{textcomp}
%\usepackage[utf8]{inputenc}
\usepackage[dvipdfmx]{graphicx}
\usepackage{comment}
\usepackage{listings, jlisting}
%\usepackage{listings-golang}
\usepackage{amsmath}

\title{電気電子計算工学及演習 課題3}
\author{三軒家 佑將(さんげんや ゆうすけ) \\ 3回生 \\ 1026-26-5817 \\ a0146089}
\date{}

\newcommand{\fg}[3]{ % \fg{label}{path}{caption},
	\begin{figure}
		\includegraphics[width=\textwidth]{#2}
		\caption{#3}
		\label{#1}
	\end{figure}
}

\newcommand{\tab}[3]{ % \tab{label}{csv-path}{caption}
	\begin{table}[htb]
  		\begin{center}
			%\input{#2}
			\csvautotabular{#2}
    		\caption{#3}
			\label{#1}
  		\end{center}
	\end{table}
}

\newcommand{\code}[2]{
	\lstinputlisting[title={ソースコード: #2(#1)}]{ ../#2 }
}

\newcommand{\fr}[1]{図\ref{#1}}
\newcommand{\tr}[1]{表\ref{#1}}
\newcommand{\er}[1]{式(\ref{#1})}

\begin{document}
    \maketitle
    
    \section{プログラムの説明}
        \subsection{概要}
            本レポートにおいては、プログラム言語としてRubyを採用した。

            プログラムを実行する手順は、以下のとおりである。以下の手順に従うことで、課題3.1, 3.2, 3.3, 3.4の4つ全てに関して、結果をグラフにした画像がgraphsディレクトリ以下に出力される。
            また、各数値解法における$p - \log_2 E_r$のグラフの傾きが、標準出力に表示される。
            \begin{lstlisting}[language=bash]
    ?> cd src
    ?> bundle install --path vendor/bin
    ?> bundle exec ruby main.rb 2> /dev/null
            \end{lstlisting}
            二行目で、依存ライブラリのインストールを行っている。また、三行目は、プログラムを実行するコマンドである。
            エラー出力を/dev/nullにリダイレクトしているのは、線形回帰に用いたライブラリの警告メッセージを表示しないためである。
            リダイレクトを行わなくても、プログラムは問題なく実行される。

        \subsection{各機能・関数の説明}
            プログラムを作成するにあたって、見通しを良くするために、プログラムを複数のファイルに分割している。
            ここでは、各ファイルごとに、そのファイルの担う機能と、そのファイル内にある関数の機能などについて簡単に説明する。

            各関数の詳しい使用方法などは、プログラム内のコメントにて示したので、そちらも参照されたい。

            \subsubsection*{calculation.rb}
                各課題の数値計算を行なう部分のうち、共通する部分を切り出したものである。calculate関数とall\_calculations関数を含む。

                calculate関数は、渡された各種パラメーターと、渡されたブロックで表されたアルゴリズムに基づいて、数値計算を行なう。

                all\_calculations関数は、渡された各種パラメーターと、渡されたブロックで表されたアルゴリズムに基づいて、calculate関数を内部で複数回呼び出し、課題3.1と3.2に示された各種数値計算を行なう。

            \subsubsection*{vector.rb}
                一次元のベクトルを表すMyVectorクラスを定義している。

                MyVectorクラスは、Rubyの組み込みクラスであるArrayクラスを継承して定義した。
                Arrayクラスの機能に加えて、ベクトル間の加算(+)・減算(-)と、ベクトル-スカラー間の乗算(*)・除算(/)を定義している。
                また、MyVectorクラスには、ベクトルの大きさ(二乗和平方根)を求めるnormメソッドと、要素の合計を求めるsumメソッドを定義した。
                さらに、MyVectorクラスのインスタンスを簡単に生成するために、Arrayクラスに、to\_vメソッドを追加した。

            \subsubsection*{plot.rb}
                グラフを描画し、ファイルに出力する機能を担う。gnuplotのラッパーを利用している。

                draw\_graphs関数に各種パラメーターを渡すことで、graphsディレクトリ以下にグラフの画像が出力される。save\_graphs関数は、draw\_graphs関数に呼び出され、実際にグラフを出力する処理を行なう。

            \subsubsection*{least\_square.rb}
                線形回帰を行って、一次関数の係数を求める機能を担う。statsampleというライブラリを利用している。

                least\_square関数が定義されており、xの配列とyの配列を与えると、その2つのデータの間に$y=a+bx$の関係があると考え、$a$と$b$の値を求める。

            \subsubsection*{main.rb}
                上記で述べた関数を利用して、実際にオイラー法・ホイン法・四次のルンゲ-クッタ法にて、微分方程式の数値解を求める。

                関数fは、与えられた微分方程式を関数にしたものである。すなわち、
                \[
                    \frac{d{\bf x}}{dt} = {\bf f} \label{eq2}
                \]
                である。

    \section{課題3.1}
        与えられた微分方程式
        \begin{eqnarray}
        \begin{split}
            \frac{d{\bf r}}{dt} & = & {\bf v}\\
            \frac{d{\bf v}}{dt} & = & \frac{q}{m}\left( {\bf v} \times {\bf B} \right)
        \end{split}
            \label{eq1}
        \end{eqnarray}
        の数値解を、オイラー法を用いて求めた。

        ただし、数値解を求める範囲は$0 \leq t \leq 6.4 * 5$とし、微小時間は${\rm dt} = 0.1$秒とした。これは、課題3.3, 3.4においても同様である。

        \subsection{オイラー法}
            オイラー法は、各時刻での各ベクトルの要素の傾きから、次の時刻での位置を計算することを繰り返す手法である。
            すなわち、ある時刻$n$での位置${\bf x}_n$は、ある微小時間${\rm dt}$を用いて、以下のように計算される。
            \begin{eqnarray*}
                {\bf x}_n = {\bf x}_{n-1} + {\bf f}({\bf x}_{n-1}) {\rm dt}
            \end{eqnarray*}
            ただし、${\bf f}$は\er{eq2}による。
        
        \subsection{結果}
            \fg{fig1}{graphs/euler/rc_ra.eps}{解析解と数値解の比較} 

            \fr{fig1}は、\er{eq1}の解析解と数値解のそれぞれについて、${\bf r}$の軌跡をプロットしたものである。
            時間の経過に従って、数値解の差が大きくなっていることがわかる。

            \fg{fig2}{graphs/euler/error_by_time.eps}{誤差の時間発展}

            また、\fr{fig2}は、\er{eq1}の解析解と数値解の誤差を、時間に沿ってプロットしたものである。
            時間の経過に従って、誤差が二次関数的に増加しているように見える。

    \section{課題3.2}
        ${\rm dt} = 6.4 * 2^{-p} (p = 3,4,..,18)$として、$0 \leq t \leq 6.4$の範囲でオイラー法による数値解を計算し、
        各$p$に対して最大誤差$E_r = \max|e_r(t)|$を求めた。

        \subsection{結果}
            \fg{fig3}{graphs/euler/error_by_p.eps}{微小時間の大きさに対する誤差の大きさ}

            \fr{fig3}は、各$p$に対する${\rm log}_2E_r$をグラフにプロットしたものである。
            $p$の増加に伴い、一次関数的に最大誤差が減少していることがわかる。

            また、このグラフの傾きは、標準出力の表示によると、
            \[
                b_e = -1.0461609925464084
            \]
            であった。

    \section{課題3.3}
        \er{eq1}の数値解を、ホイン法を用いて計算した。
        また、課題3.2と同様に、各$p$に対して最大誤差$E_r = \max|e_r(t)|$を求めた。

        \subsection{ホイン法}
            ホイン法は、以下の手順で${\bf x}_{n}$を順次求める手法である。

            \begin{enumerate}
                \item オイラー法により${\bf x}_{n-1}$から、仮の${\bf x}_{n}$を求める。
                \item ${\bf k} = \frac{{\bf f}({\bf x}_{n-1}) + {\bf f}({\bf x}_{n})}{2}$とする。
                \item ${\bf x}_{n} = {\bf x}_{n-1} + {\bf k} * {\rm dt}$とする。
            \end{enumerate}

        \subsection{結果}
            \fg{fig4}{graphs/heun/rc_ra.eps}{解析解と数値解の比較} 

            \fr{fig4}は、\er{eq1}の解析解と数値解のそれぞれについて、${\bf r}$の軌跡をプロットしたものである。
            目視では違いが見られないほど、高い精度で数値解が求められていることがわかる。

            \fg{fig5}{graphs/heun/error_by_time.eps}{誤差の時間発展}

            また、\fr{fig5}は、\er{eq1}の解析解と数値解の誤差を、時間に沿ってプロットしたものである。
            時間の経過に従って、誤差が一次関数的に増加していることがわかる。

            \fg{fig6}{graphs/heun/error_by_p.eps}{微小時間の大きさに対する誤差の大きさ}

            \fr{fig6}は、各$p$に対する${\rm log}_2E_r$をグラフにプロットしたものである。
            $p$の増加に伴い、一次関数的に最大誤差が減少していることがわかる。

            また、このグラフの傾きは、標準出力の表示によると、
            \[
                b_e = -1.9974708704503414
            \]
            であった。

    \section{課題3.4}
        \er{eq1}の数値解を、4次のルンゲ-クッタ法を用いて計算した。
        また、課題3.2と同様に、各$p$に対して最大誤差$E_r = \max|e_r(t)|$を求めた。

        \subsection{4次のルンゲ-クッタ法}
            4次のルンゲ-クッタ法は、以下の手順で${\bf x}_{n}$を順次求める手法である。

            \begin{enumerate}
                \item オイラー法により${\bf x}_{n-1}$から、仮の${\bf x}_{n}$を求める。
                \item ${\bf k} = \frac{{\bf f}({\bf x}_{n-1}) + {\bf f}({\bf x}_{n})}{2}$とする。
                \item ${\bf x}_{n} = {\bf x}_{n-1} + {\bf k} * {\rm dt}$とする。
            \end{enumerate}
        % 結果(グラフを貼り付ける、傾きに関しては標準出力からの出力であることを明記する)

    \section{考察}
        % 考察(計算方法に関する考察、得られた結果の精度、うまく行かなかったと思われる点、その他、気付いた点や考えたことについて)

    \section{付録}
        % 実行結果のコピー

    \section{参考文献}
        \begin{enumerate}
            \item オイラー法 - Wikipedia (https://goo.gl/sKVLx1)
            \item Heun法 - [物理のかぎしっぽ] (https://goo.gl/0DH44q)
            \item Runge-Kutta法 - [物理のかぎしっぽ] (https://goo.gl/raIx64)
        \end{enumerate}
\end{document}
