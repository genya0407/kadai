\documentclass[11pt]{jsarticle}

\usepackage{color}
\usepackage{xcolor}
\usepackage{textcomp}
\usepackage[utf8]{inputenc}
\usepackage{graphicx}
\usepackage{comment}
\usepackage{listings, jlisting}
\usepackage{listings-golang}

\lstset{ % add your own preferences
    frame=single,
    basicstyle=\footnotesize\ttfamily,
    keywordstyle=\color{red},
    numbers=left,
    numbersep=5pt,
    showstringspaces=false, 
    stringstyle=\color{blue},
    tabsize=4,
    language=Golang, % this is it !
	breaklines=true
}

\title{電気電子計算工学及演習}
\author{三軒家 佑將(さんげんや ゆうすけ) \\ 3回生 \\ 1026-26-5817 \\ a0146089}
\date{}

\newcommand{\fg}[3]{ % \fg{label}{path}{caption},
	\begin{figure}
		\includegraphics[width=\textwidth]{#2}
		\caption{#3}
		\label{#1}
	\end{figure}
}

\newcommand{\tab}[3]{ % \tab{label}{csv-path}{caption}
	\begin{table}[htb]
  		\begin{center}
			%\input{#2}
			\csvautotabular{#2}
    		\caption{#3}
			\label{#1}
  		\end{center}
	\end{table}
}

\newcommand{\code}[2]{
	\lstinputlisting[title={ソースコード: #2(#1)}]{ ../#2 }
}

\newcommand{\fr}[1]{図\ref{#1}}
\newcommand{\tr}[1]{表\ref{#1}}

\begin{document}
    \maketitle

    \section{前進代入}
    	\subsection{採用したアルゴリズム}
			\[
				y_i = b_i - \sum_{k=0}^{i-1}l_{ik}y_k
			\]
			として、$i=0,1,2$の順に$y_i$を求めた。

		\subsection{プログラムに関する情報}
			\begin{description}
	        	\item[ファイル名] \mbox{}
					\begin{itemize}
						\item 1.go
						\item forward.go
						\item print.go
					\end{itemize}
				\item[コンパイルコマンド] \mbox{} \\
					go run 1.go print.go forward.go
			\end{description}

			\subsubsection*{作成した主な関数}
				\begin{description}
					\item[PrintVector] \mbox{}\\
						ベクトル(1次元配列)を表示する関数。引数としてベクトルを渡す。
					\item[PrintMatrix] \mbox{}\\
						行列(2次元配列)を表示する関数。引数としてベクトルを渡す。
					\item[Forward] \mbox{}\\
						前進代入法によって方程式の解を求める関数。
						第一引数として行列(下三角行列)を、第二引数としてベクトル($b=Lx$のときの$b$)を渡すと、方程式の解をベクトルとして返す。
				\end{description}

		\subsection{結果}
			手計算の結果は、
			\[
				{\bf y} = \left(
					\begin{array}{c}
						9 \\
						8 \\
						-4
					\end{array}
				\right)
			\]
			であった。また、プログラムによる数値解は、
			\[
				{\bf y} = \left(
					\begin{array}{c}
						9.000 \\
						8.000 \\
						-4.000
					\end{array}
				\right)
			\]
			であった。
		\subsection{考察}
			手計算による解とプログラムによる数値解は一致していた。

    \section{後退代入}
    	\subsection{採用したアルゴリズム}
			\[
				x_i = \frac{1}{u_{ii}}\left( c_i - \sum_{k=i+1}^{n-1} u_{ik}x_k \right)
			\]
			として、$i=2,1,0$の順に$x_i$を求めた。

		\subsection{プログラムに関する情報}
			\begin{description}
	        	\item[ファイル名]\mbox{}
					\begin{itemize}
						\item 2.go
						\item backward.go
					\end{itemize}
				\item[コンパイルコマンド] \mbox{}\\
					go run 2.go print.go backward.go
			\end{description}

			\subsubsection*{作成した主な関数}
				\begin{description}
					\item[Backward] \mbox{}\\
						後退代入法によって方程式の解を求める関数。
						第一引数として行列(上三角行列)を、第二引数としてベクトル($b=Ux$のときの$b$)を渡すと、方程式の解をベクトルとして返す。
				\end{description}

		\subsection{結果}
			手計算の結果は、
			\[
				{\bf y} = \left(
					\begin{array}{c}
						2 \\
						-3 \\
						-2
					\end{array}
				\right)
			\]
			であった。また、プログラムによる数値解は、
			\[
				{\bf y} = \left(
					\begin{array}{c}
						2.000 \\
						-3.000 \\
						-2.000
					\end{array}
				\right)
			\]
			であった。
		\subsection{考察}
			手計算による解とプログラムによる数値解は一致していた。

    \section{行列の計算}
		\subsection{結果}
			手計算により、
			\begin{equation}
				\left(
					\begin{array}{ccc}
						a_{00} & a_{01} & a_{02} \\
						a_{10} & a_{11} & a_{12} \\
						a_{20} & a_{21} & a_{22}
					\end{array}
				\right)
					=
				\left(
					\begin{array}{ccc}
						3 & 1 & -3 \\
						-3 & -3 & 2 \\
						-6 & -8 & 5
					\end{array}
				\right) \label{eq1}
			\end{equation}
			と求められた。

    \section{LU分解}
    	\subsection{採用したアルゴリズム}
			\begin{eqnarray*}
				u_{ij} &=& a_{ij} - \sum_{k=0}^{i-1} \\
				l_{ij} &=& \frac{1}{u_{jj}}\left( a_{ij} - \sum_{k=0}^{j-1} l_{jk}u{kj} \right)
			\end{eqnarray*}
			として、$u_{ij}, l_{ij}$を求めた。
			
			このとき、問に与えられた順序に従って計算するのは面倒なので、添字から$u_{ij}, l_{ij}$を計算する関数をそれぞれ用意し、内部的に再帰呼び出しするように実装した。
			
		\subsection{プログラムに関する情報}
			\begin{description}
	        	\item[ファイル名]\mbox{}
					\begin{itemize}
						\item 4.go
						\item lu.go
					\end{itemize}
				\item[コンパイルコマンド]\mbox{}\\
					go run 4.go print.go lu.go
			\end{description}

			\subsubsection*{作成した主な関数}
				\begin{description}
					\item[Decomp] \mbox{}\\
						matrix.Decomp\(\)のように呼び出すと、上三角行列と下三角行列を返す関数。
						内部的にlower関数とupper関数を呼び出し、2次元配列に入れて、それを返すだけの関数。
					\item[upper] \mbox{}\\
						matrix.upper(i,j)のように呼び出すと、添字($i$と$j$)に対応する上三角行列の要素を返す関数。
						内部的にlower関数を呼び出している。
					\item[lower] \mbox{}\\
						matrix.upper(i,j)のように呼び出すと、添字($i$と$j$)に対応する下三角行列の要素を返す関数。
						内部的にupper関数を呼び出している。
				\end{description}

		\subsection{結果}
			式(\ref{})をDecomp関数によりLU分解した結果、
			\begin{eqnarray*}
				L &=& \left(
					\begin{array}{ccc}
						1.000 & 0.000 & 0.000 \\
						-1.000 & 1.000 & 0.000 \\
						-2.000 & 3.000 & 1.000
					\end{array}
				\right) \\
				U &=& \left(
					\begin{array}{ccc}
						3.000 & 1.000 & -3.000 \\
						0.000 & -2.000 & -1.000 \\
						0.000 & 0.000 & 2.000
					\end{array}
				\right)
			\end{eqnarray*}
			となった。
		\subsection{考察}
			LU分解の結果は、式(\ref{eq1})の計算前の式と一致しているため、妥当であると考えられる。
			
			今回作成したプログラムは、lower関数とupper関数が呼び出されるたびに、大量の計算が発生する。
			$N=11$程度まではすぐに計算が終了したが、それより$N$が大きくなると、プログラムはなかなか終了しなくなった。
			$N=12$以上のケースに対応するには、メモ化が必要である。
			
			この場合は、lower関数とupper関数の計算結果を、それぞれ「添字→計算結果」のハッシュに格納し、
			すでに計算済みの要素に関しては、ハッシュに格納されている値を返すようにする。
			これにより、一度計算した要素を計算し直すことがなくなり、計算量は大幅に減る。
			
			実際、問8のプログラムの実行に、メモ化なしの場合は$N=12$のとき25秒ほどかかっていたが、メモ化をした場合は$N=50$のときでも10秒ほどで終了した。

    \section{$n$元連立一次方程式の解法}
    	\subsection{採用したアルゴリズム}
			以下の手順により、$n$元連立1次方程式${\bf Ax} = {\bf b}$の解${\bf x}$を求める。
			
			まず、
			\[
				{\bf A} = {\bf LU}
			\]
			のようにLU分解する。
			
			次に、
			\begin{equation}
				{\bf Ux} = {\bf y} \label{eq2}
			\end{equation}
			と置く。
			
			これにより、
			\[
				{\bf Ly} = {\bf b}
			\]
			が成立する。これを、前進代入法によって、${\bf y}$について解く。
			
			最後に、この${\bf y}$の値を用いて、式(\ref{eq2})を${\bf x}$について解く。

		\subsection{プログラムに関する情報}
			\begin{description}
	        	\item[ファイル名]\mbox{}
					\begin{itemize}
						\item 5.go
						\item solve.go
					\end{itemize}
				\item[コンパイルコマンド]\mbox{}\\
					go run 5.go print.go solve.go lu.go forward.go backward.go
			\end{description}

			\subsubsection*{作成した主な関数}
				\begin{description}
					\item[Solve] \mbox{}\\
						第一引数として行列${\bf A}$を、第二引数としてベクトル{\bf b}を渡すと、${\bf Ax} = {\bf b}$の解ベクトル${\bf x}$を返す関数。
						内部では、問4にて作成したDecomp関数、問1で作成したForward関数、問2で作成したBackward関数を利用している。
				\end{description}

		\subsection{結果}
			以下の3元連立一次方程式の解をプログラムにより計算した。
			\[
				\left(
					\begin{array}{ccc}
						3 & 1 & -3 \\
						-3 & -3 & 2 \\
						-6 & -8 & 5
					\end{array}
				\right)
				{\bf x}
					=
				\left(
					\begin{array}{ccc}
						9.0\\
						-1.0\\
						2.0
					\end{array}
				\right)
			\]
			その結果は、
			\[
				{\bf x} =
					\left(
						\begin{array}{ccc}
							2.000\\
							-3.000\\
							-2.000
						\end{array}
					\right)
			\]
			となった。
		
		\subsection{考察}
			上記で求めた解を、問の式に代入して計算すると、
			\[
				\left(
					\begin{array}{ccc}
						3 & 1 & -3 \\
						-3 & -3 & 2 \\
						-6 & -8 & 5
					\end{array}
				\right)
				\left(
					\begin{array}{ccc}
						2.000\\
						-3.000\\
						-2.000
					\end{array}
				\right)
					=
				\left(
					\begin{array}{ccc}
						9.0\\
						-1.0\\
						2.0
					\end{array}
				\right)
			\]
			となり、確かに解である事がわかる。

    \section{$n$元連立一次方程式の解を求める}
    	\subsection{採用したアルゴリズム}
			問5と同様のアルゴリズムを採用する。

		\subsection{プログラムに関する情報}
			6\_1.go及び6\_2.goはNの値を与えているだけである。本質的な実装は6.goにある。

			\begin{description}
	        	\item[ファイル名]\mbox{}
					\begin{itemize}
						\item 6\_1.go
						\item 6\_2.go
						\item 6.go
					\end{itemize}
				\item[コンパイルコマンド]\mbox{}\\
					\begin{description}
						\item [$N=3$の場合] go run 6\_1.go print.go 6.go solve.go lu.go forward.go backward.go
						\item [$N=6$の場合] go run 6\_2.go print.go 6.go solve.go lu.go forward.go backward.go
					\end{description}
			\end{description}
		
			\subsubsection*{作成した主な関数}
				\begin{description}
					\item[main] \mbox{}\\
						内部で行列${\bf H}$とベクトル${\bf b}$を計算している。
				\end{description}

		\subsection{結果}
			$h_{ij} = 0.25^{|i-j|}$を要素とした行列${\bf H}$と、$b_i = 5.0 - 4.0^{i-n+1} - 4.0^{-i}$を要素とするベクトル${\bf b}$を用いて、
			\begin{equation}
				{\bf Hx} = {\bf b} \label{eq3}
			\end{equation}
			と表されるN元連立一次方程式の数値解${\bf x}$を、$N=3,6$の場合についてそれぞれプログラムで求めた。
			
			その結果は、$N=3$のとき、
			\[
				{\bf x} =
				\left(
					\begin{array}{c}
						3.000\\
						3.000\\
						3.000
					\end{array}
				\right)
			\]
			となり、$N=6$のとき、
			\[
				{\bf x} =
				\left(
					\begin{array}{c}
						3.000\\
						3.000\\
						3.000\\
						3.000\\
						3.000\\
						3.000
					\end{array}
				\right)
			\]
			となった。

		\subsection{考察}
			式(\ref{eq3})の解は、要素がすべて3のベクトルになるので、得られた数値解はどちらも妥当であると考えられる。

    \section{逆行列を求めるアルゴリズム}
    	\subsection{採用したアルゴリズム}
			\begin{eqnarray*}
				{\bf b_0} &=& \left(
					\begin{array}{c}
						1\\
						0\\
						0
					\end{array}
				\right)\\
				{\bf b_1} &=& \left(
					\begin{array}{c}
						0\\
						1\\
						0
					\end{array}
				\right)\\
				{\bf b_2} &=& \left(
					\begin{array}{c}
						0\\
						0\\
						1
					\end{array}
				\right)
			\end{eqnarray*}
			として、
			\begin{eqnarray*}
				{\bf Ax_0} &=& {\bf b_0}\\
				{\bf Ax_1} &=& {\bf b_1}\\
				{\bf Ax_2} &=& {\bf b_2}\\
			\end{eqnarray*}
			を満たす${\bf x_0,x_1,x_2}$を求める。このとき、
			\[
				{\bf A^{-1} = \left( 
					\begin{array}{ccc}
						x_0 & x_1 & x_2
					\end{array}
				\right)}
			\]
			によって、$A^{-1}$を求める。

		\subsection{プログラムに関する情報}
			\begin{description}
	        	\item[ファイル名]\mbox{}
					\begin{itemize}
						\item inverse.go
					\end{itemize}
			\end{description}

			\subsubsection*{作成した主な関数}
				\begin{description}
					\item[Inverse] \mbox{}\\
						引数として行列を与えると、その逆行列を返す関数。
					\item[setCol] \mbox{}\\
						matrix.setCol(j, vector)などと実行する。\\
						第一引数には列番号を、第二引数にはその列に代入する列ベクトルを与える。\\
						この関数を実行すると、呼び出し元の行列(上の例ではmatrix変数)の、j列目が、vectorになる。
					\item[idMatrix] \mbox{}\\
						引数無しで実行され、N行N列の単位行列を返す関数。
				\end{description}

    \section{逆行列を計算する}
    	\subsection{採用したアルゴリズム}
			問7で示したアルゴリズムを採用して逆行列を求める。

		\subsection{プログラムに関する情報}
			8\_1.go及び8\_2.goはNの値を与えているだけである。本質的な実装は8.goにある。

			また、出力される行列はそれぞれ、
			\begin{eqnarray*}
				I1 & = & HH^{-1}\\
				I2 & = & H^{-1}H
			\end{eqnarray*}
			を示している。

			\begin{description}
	        	\item[ファイル名]\mbox{}
					\begin{itemize}
						\item 8\_1.go
						\item 8\_2.go
						\item 8.go
					\end{itemize}
				\item[コンパイルコマンド]\mbox{}
					\begin{description}
						\item [$N=3$の場合]\mbox{}\\
							go run 8\_1.go print.go 8.go inverse.go solve.go lu.go forward.go backward.go
						\item [$N=6$の場合]\mbox{}\\
							go run 8\_2.go print.go 8.go inverse.go solve.go lu.go forward.go backward.go
					\end{description}
			\end{description}

			\subsubsection*{作成した主な関数}
				\begin{description}
					\item[MatMul] \mbox{}\\
						第一引数と第二引数に行列をとり、その2つの行列の行列積を返す関数。
				\end{description}

		\subsection{結果}
			プログラムによるI1,I2の計算結果は以下のようであった。
			
			$N=3$のとき、
			\begin{eqnarray*}
				I1 & = & \left( \begin{array}{ccc}
					1.000 & -0.000 & 0.000\\
					0.000 & 1.000 & 0.000\\
					0.000 & 0.000 & 1.000
				\end{array} \right)\\
				I2 & = & \left( \begin{array}{ccc}
					1.000 & 0.000 & 0.000\\
					-0.000 & 1.000 & 0.000\\
					0.000 & 0.000 & 1.000
				\end{array} \right)
			\end{eqnarray*}
			
			$N=6$のとき、
			\begin{eqnarray*}
				I1 & = & \left( \begin{array}{cccccc}
					1.000 & -0.000 & -0.000 & -0.000 & -0.000 & 0.000\\
					0.000 & 1.000 & -0.000 & -0.000 & -0.000 & 0.000\\
					0.000 & 0.000 & 1.000 & -0.000 & -0.000 & 0.000\\
					0.000 & 0.000 & 0.000 & 1.000 & -0.000 & 0.000\\
					0.000 & 0.000 & 0.000 & 0.000 & 1.000 & 0.000\\
					0.000 & 0.000 & 0.000 & 0.000 & 0.000 & 1.000
				\end{array} \right)\\
				I2 & = & \left( \begin{array}{cccccc}
					1.000 & 0.000 & 0.000 & 0.000 & 0.000 & 0.000\\
					-0.000 & 1.000 & 0.000 & 0.000 & 0.000 & 0.000\\
					-0.000 & -0.000 & 1.000 & 0.000 & 0.000 & 0.000\\
					-0.000 & -0.000 & -0.000 & 1.000 & 0.000 & 0.000\\
					-0.000 & -0.000 & -0.000 & -0.000 & 1.000 & 0.000\\
					0.000 & 0.000 & 0.000 & 0.000 & 0.000 & 1.000
				\end{array} \right)
			\end{eqnarray*}
			
		\subsection{考察}
			$N=3,N=6$のいずれの場合についても、$I1,I2$は、「対角成分は1.000、それ以外の要素は0.000または-0.000」の行列となった。
			-0.000となっている要素は、計算の過程で生まれた誤差によって、絶対値のごく小さい負の値をとっていると考えられる。

	\newpage
	\section*{付録}
		\subsection*{ソースコード}
			作成したプログラムを以下に示す
			\code{共通}{print.go}
			\code{問1}{1.go}
			\code{問1}{forward.go}
			\code{問2}{2.go}
			\code{問2}{backward.go}
			\code{問4}{lu.go}
			\code{問4}{solve.go}
			\code{問5}{5.go}
			\code{問6}{"6\string_1.go"}
			\code{問6}{"6\string_2.go"}
			\code{問6}{6.go}
			\code{問8}{"8\string_1.go"}
			\code{問8}{"8\string_2.go"}
			\code{問8}{8.go}
			\code{問8}{inverse.go}
		\newpage

		\subsection*{プログラムの実行結果}
			すべてのプログラムを実行するshellscriptと、その出力結果を示す。
			\lstinputlisting[title={shellscript}]{ ../commands.sh }
			\lstinputlisting[title={出力結果}]{ ../output.log }

\end{document}

\begin{comment}

    \section{}
    	\subsection{採用したアルゴリズム}
		\subsection{プログラムに関する情報}
			\begin{description}
	        	\item[ファイル名]\mbox{}
					\begin{itemize}
						\item 
					\end{itemize}
				\item[コンパイルコマンド]\mbox{}\\
			\end{description}
			\subsubsection*{作成した主な関数}
				\begin{description}
					\item[名前] \mbox{}\\
						機能
				\end{description}
		\subsection{結果}
		\subsection{考察}

\end{comment}

