\documentclass[11pt]{ltjsarticle}

\usepackage{fontspec}
\usepackage{luatexja-fontspec}
\usepackage[hiragino-pron]{luatexja-preset}
\usepackage{luatexja-ruby}
\usepackage{graphicx}
\usepackage{comment}

\title{電気電子計算工学及演習}
\author{三軒家 佑將(さんげんや ゆうすけ) \\ 3回生 \\ 1026-26-5817 \\ a0146089}
\date{}

\newcommand{\fg}[3]{ % \fg{label}{path}{caption},
	\begin{figure}
		\includegraphics[width=\textwidth]{#2}
		\caption{#3}
		\label{#1}
	\end{figure}
}

\newcommand{\tab}[3]{ % \tab{label}{csv-path}{caption}
	\begin{table}[htb]
  		\begin{center}
			%\input{#2}
			\csvautotabular{#2}
    		\caption{#3}
			\label{#1}
  		\end{center}
	\end{table}
}

\newcommand{\fr}[1]{図\ref{#1}}
\newcommand{\tr}[1]{表\ref{#1}}

\begin{comment}

    \section{}
    	\subsection{採用したアルゴリズム}
		\subsection{プログラムに関する情報}
			\begin{description}
	        	\item[ファイル名]\mbox{}
					\begin{itemize}
						\item
					\end{itemize}
				\item[コンパイルコマンド]\mbox{}\\
			\end{description}
			\subsubsection*{作成した主な関数}
				\begin{description}
					\item[名前] \mbox{}\\
						機能
				\end{description}
		\subsection{結果}
		\subsection{考察}

\end{comment}

\begin{document}
    \maketitle

    \section{前進代入}
    	\subsection{採用したアルゴリズム}
			\[
				y_i = b_i - \sum_{k=0}^{i-1}l_{ik}y_k
			\]
			として、$i=0,1,2$の順に$y_i$を求めた。

		\subsection{プログラムに関する情報}
			\begin{description}
	        	\item[ファイル名] \mbox{}
					\begin{itemize}
						\item 1.go
						\item forward.go
						\item print.go
					\end{itemize}
				\item[コンパイルコマンド] \mbox{} \\
					go run 1.go print.go forward.go
			\end{description}

			\subsubsection*{作成した主な関数}
				\begin{description}
					\item[PrintVector] \mbox{}\\
						ベクトル(1次元配列)を表示する関数。引数としてベクトルを渡す。
					\item[PrintMatrix] \mbox{}\\
						行列(2次元配列)を表示する関数。引数としてベクトルを渡す。
					\item[Forward] \mbox{}\\
						前進代入法によって方程式の解を求める関数。
						第一引数として行列(下三角行列)を、第二引数としてベクトル($b=Lx$のときの$b$)を渡すと、方程式の解をベクトルとして返す。
				\end{description}

		\subsection{結果}
			手計算の結果は、
			\[
				{\b y} = \left(
					\begin{array}{c}
						9 \\
						8 \\
						-4
					\end{array}
				\right)
			\]
			であった。また、プログラムによる数値解は、
			\[
				{\b y} = \left(
					\begin{array}{c}
						9.000 \\
						8.000 \\
						-4.000
					\end{array}
				\right)
			\]
			であった。
		\subsection{考察}
			手計算による解とプログラムによる数値解は一致していた。

    \section{後退代入}
    	\subsection{採用したアルゴリズム}
			\[
				x_i = \frac{1}{u_{ii}}\left( c_i - \sum_{k=i+1}^{n-1} u_{ik}x_k \right)
			\]
			として、$i=2,1,0$の順に$x_i$を求めた。

		\subsection{プログラムに関する情報}
			\begin{description}
	        	\item[ファイル名]\mbox{}
					\begin{itemize}
						\item 2.go
						\item backward.go
					\end{itemize}
				\item[コンパイルコマンド] \mbox{}\\
					go run 2.go print.go backward.go
			\end{description}

			\subsubsection*{作成した主な関数}
				\begin{description}
					\item[Backward] \mbox{}\\
						後退代入法によって方程式の解を求める関数。
						第一引数として行列(上三角行列)を、第二引数としてベクトル($b=Ux$のときの$b$)を渡すと、方程式の解をベクトルとして返す。
				\end{description}

		\subsection{結果}
			手計算の結果は、
			\[
				{\b y} = \left(
					\begin{array}{c}
						2 \\
						-3 \\
						-2
					\end{array}
				\right)
			\]
			であった。また、プログラムによる数値解は、
			\[
				{\b y} = \left(
					\begin{array}{c}
						2.000 \\
						-3.000 \\
						-2.000
					\end{array}
				\right)
			\]
			であった。
		\subsection{考察}
			手計算による解とプログラムによる数値解は一致していた。
\end{document}
