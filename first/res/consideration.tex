\section{1-4の考察}

$f1$は、$|f(x_{k})| \lt \epsilon$も$|x_{k} - x_{k-1}| \lt \epsilon$も満たして終了している。

$f2$は、$|x_{k} - x_{k-1}| \lt \epsilon$は満たされているが、$|f(x_{k})| \lt \epsilon$は満たされていない。何分か実行し続けたが、終了しなかった。

これは、$f2$の中には指数関数が入っているため、$x$の小さな変化に対しても、鋭敏に$f(x)$の値が変化してしまうことが原因であると考えられる。

$|x_{k} - x_{k-1}| \lt \epsilon$が成立する$x_{k}$の近傍での$e^{(x+1)}$の一次導関数の値は、$e^{x+1} \approx 37 $であり、
float64型の誤差は10進数でおおよそ$10^{-16}$程度であるので、$f(x)$の誤差は、少なくとも$37 \times 10^{-16} \gt 10^{-15}$程度は発生することになるため、
必ずしも$|f(x)| \lt 10^{-15}$が満たされるとは限らない。

実際、$|x_{k} - x_{k-1}| \lt \epsilon$が満たされたときの$|f(x)|$の値は、$10^(-15)$程度の値であり、ある程度の値には収束していることがわかる。
